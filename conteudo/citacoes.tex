\chapter{Citações}

Há duas formas de se fazer uma citação: a \emph{citação indireta ou livre} (também chamada de \emph{paráfrase}) e a \emph{citação direta ou textual}. Pode haver, ainda, a \emph{citação de citação}.

Todas as citações devem trazer a \emph{identificação} de sua autoria.

\section{Citação indireta ou livre (paráfrase)}

Chamamos de citação indireta ou livre (\emph{paráfrase}) aquela citação na qual expressamos o \emph{pensamento de outra pessoa} com \emph{nossas próprias palavras}.

Após fazermos a citação, devemos indicar o nome do autor, \emph{em letras minúsculas}, se estiver no corpo do texto, e com letras \emph{maiúsculas}, se estiver dentro dos parênteses, juntamente com o \emph{ano} da publicação da obra em que se encontra a idéia por nós referida. Não são indicadas páginas já que a idéia pode estar sendo resumida de uma obra inteira, de um capítulo, de diversas partes ou de um conjunto delas.

Desta forma (com o nome no corpo do texto):

Depois de analisar a situação, \citeonline{novoa1993} chegou a afirmar que o brasileiro ainda não está capacitado para escolher seus governantes por causa de sua precária vocação política e da absoluta falta de escolaridade, já que o homem do povo, o zé-povinho, geralmente não sabe sequer em quem votou nas últimas eleições, não sabe sequer quem são seus governantes, não saber sequer quem determina seu próprio meio de sobreviver.

Ou, então, (com o nome nos parênteses):

Depois de analisar a situação, chegou-se a afirmar que o brasileiro ainda não está capacitado para escolher seus governantes por causa de sua precária vocação política e da absoluta falta de escolaridade, já que o homem do povo, o zé-povinho, geralmente não sabe sequer em quem votou nas últimas eleições, não sabe sequer quem são seus governantes, não saber sequer quem determina seu próprio meio de sobreviver \cite{novoa1993}.

No caso de o autor possuir outras obras, elas serão diferenciadas pela data da publicação. Havendo mais de uma obra no mesmo ano, acrescentamos uma letra após a data.

No caso do teatro ou do cinema quem melhor se definiu foi \citeonline{antunes1997a} quando declarou que aqueles espaços haviam sido todos tomados pela geração de 40. Por outro lado, ele próprio se contradisse, mais tarde, \citeonline{antunes1997b}, como já se contradissera noutras ocasiões, ao referir-se às decisões tomadas pelos autores da geração de 50. Isso é uma incongruência com a qual convivemos há muito tempo.

Quando, no transcorrer do texto, em citações indiretas ou livres, se faz menção, seguidas vezes, ao mesmo autor, na mesma obra, não é necessário que se repita a indicação do ano.

\begin{table}[h]
    \caption{ Deve-se escolher somente um tipo de citação para usar durante o texto}
    \begin{center}
        \begin{tabular}{ c | p{0.6\textwidth} }
            \hline
            \multicolumn{2}{ c }{FORMATAÇÃO DAS CITAÇÕES DOS AUTORES DURANTE O TEXTO} \\
            \hline
            \citeonline{novoa1993} & O nome do autor deve ser escrito em letras \emph{minúsculas} quando apresentado no próprio texto \\
            \hline
            \cite[p. 32]{guimaraes1985} & O nome do autor deve ser escrito em letras \emph{maiúsculas} quando apresentado dentro dos parênteses. \\
            \hline
        \end{tabular}
    \end{center}
    \legend{Fonte: \cite[p. 356]{meregali2004}}
    \label{tab:tipos-citacao}
\end{table}

\section{Citação direta ou textual (transcrição)}

São chamadas de citações diretas ou textuais aquelas em que se transcrevem \emph{exatamente as palavras do autor citado}. As citações diretas ou textuais podem ser \emph{breves} ou \emph{longas}.

São consideradas \emph{breves} aquelas cuja extensão não ultrapassa \emph{três linhas}. Essas citações devem \emph{integrar o texto} e devem vir \emph{entre aspas}. \emph{O tamanho} da \emph{fonte} (letra) da citação breve \emph{permanece} o mesmo do corpo do texto (\emph{pitch 12}).

Vimos que, para nosso esclarecimento, precisamos seguir os preceitos encontrados, já que Guimarães estabelece: ``A valorização da palavra pela palavra encarna o objetivo precípuo do texto literário'' (1985, p. 32) e, se isso não ficar bem esclarecido, nosso trabalho será seriamente prejudicado.

Ou assim:

Vimos que, para nosso esclarecimento, precisamos seguir os preceitos encontrados, já que ficou estabelecido que ``a valorização da palavra pela palavra encarna o objetivo precípuo do texto literário'' \cite[p. 32]{guimaraes1985} e, se isso não ficar bem esclarecido, nosso trabalho será seriamente prejudicado.

As citações com mais de três linhas são chamadas de \emph{longas} e devem receber um destaque especial com recuo (reentrada) de \emph{4cm} ou \emph{dezesseis toques}, da margem, mais \emph{cinco} toques para o início do parágrafo.

As citações longas, por já terem o destaque do recuo (reentrada), \emph{não deverão ter aspas} e o tamanho da fonte (letra) deve ser \emph{menor} que o do texto: \emph{pitch 10}.

A distância entre as linhas do corpo da citação deve ser de um espaço \emph{simples}. Entre o texto da citação e o restante do trabalho, deve-se deixar \emph{dois espaços duplos}, antes e depois.

Há uma certa dificuldade quanto ao reconhecimento de \emph{O, A, OS, AS} como pronomes demonstrativos, mas essa dúvida é muito bem dirimida por Fernandes:

\begin{quote}Os pronomes O, A, OS e AS passam a ser pronomes demonstrativos sempre que numa frase puderem ser substituídos, sem alterar a estrutura dessa frase, respectivamente, por ISTO, ISSO, AQUILO, AQUELE, AQUELES, AQUELA, AQUELAS (1994, p. 19.).\end{quote}

Havendo \emph{supressão} de trechos \emph{dentro do texto} citado, faz-se a indicação com reticências entre colchetes [\ldots]:

``Na comunicação diária, aquela comunicação que utilizamos no dia-a-dia, junto de nossos familiares e amigos, por exemplo, além da referencialidade da linguagem [\ldots] há pinceladas de função conativa'' \cite[p. 37]{chalhub1991}.

No \emph{início} ou no \emph{fim} da citação, as reticências são usadas apenas quando o trecho citado \emph{não é uma sentença completa}. Entende-se por sentença completa aquela que o autor elaborou, com todos os seus elementos, isto é, uma sentença que contenha sujeito, predicado e seus complementos gramaticais exigidos. Caso contrário, \emph{se a sentença for completa}, no início ou no termino de citação, \emph{não se deve fazer} o uso das reticências. \emph{É óbvio} que se trata de parte de um todo, que se retirou um trecho, portanto, não há necessidade de se indicar com as reticências.

Encerrava seu discurso nomeando os que figurariam somente nos exercício gerais, citando palavras de ordem, dentre as quais pudemos entender:

``\ldots muitas mortes, desaparecimentos e desolação haverão de varrer este pais de norte a sul, de lesta a oeste e nada restará para a posteridade que sentirá a falta de um elo'' \cite{morgado1967}.

Mais adiante, aquilo que mais chocou a todos quanto o ouviam:

``Arrasem com tudo, queimem tudo, ponham tudo abaixo, destruam com tudo, não poupem ninguém, nem crianças, nem mulheres, nem velhos\ldots'' \cite{morgado1967}.

Se a citação for usada para completar uma sentença do autor do Trabalho, esta terminará em vírgula e aquela iniciará \emph{sem a entrada de parágrafo} e \emph{com letra minúscula}.

A secretária ameaçou, dizendo que, ``da próxima vez, a máquina ficará sem as peças de reposição, se ele não chegar e disser o que precisa ser dito, uma vez que não estou aqui para servir de adivinha para seus caprichos desencontrados e sem nexo.'' \cite[p. 34]{marques1982}.

Caso o texto do autor do Trabalho seja uma \emph{continuação} da citação, esta \emph{terminará por vírgula} e o texto reiniciado \emph{sem entrada de parágrafo e com letra minúscula}.

 Os gramáticos são claros quando assumem uma posição quanto ao emprego do pronome oblíquo no início de oração. \citeonline[p. 419]{cegalla1991} diz claramente que: \begin{quote}Iniciar a frase com o pronome átono só é lícito na conversação familiar, despreocupada, ou na língua escrita, quando se deseja reproduzir a fala dos personagens, porém nós sabemos que na prática não é bem assim que acontece - as normas, rigorosamente, são esquecidas por quase todos os usuários do idioma falado, principalmente nas ocasiões informais.\end{quote}

Quando houver uma citação \emph{dentro de outra citação}, as aspas da segunda transformam-se em aspas simples (` e ') (apóstrofo: Não confundir a palavra \emph{apóstrofo} que é o sinal (` e '), com \emph{apóstrofe} que é uma figura de linguagem que consiste na interpelação ou invocação do leitor, ouvinte ou outra pessoa no decorrer de um texto). Quando dentro da citação transcrita houver aspas, estas também são mudadas para aspas simples.

Se for feita alguma \emph{interpelação}, \emph{acréscimo} ou \emph{comentário} durante a citação, deve-se fazê-lo \emph{entre colchetes} [ ]:

Também chamado de corpo do trabalho, [o desenvolvimento] tem por finalidade expor, demonstrar e fundamentar a explicitação do assunto a ser abordado. É normalmente dividido em seções ou capítulos, que variam de acordo com a natureza do assunto. \cite[p. 17]{garcia2000}.

Se algum \emph{destaque} (grifo, negrito, itálico ou sublinhado) for dado, deve-se indicá-lo com a expressão \emph{grifo nosso}, entre colchetes:

A primeira citação de uma obra deve ter sua referência bibliográfica completa. As subseqüentes citações da mesma obra \emph{podem ser referendadas de forma abreviada}, desde que não haja referências intercaladas de outras obras do mesmo autor (NBR 6023-2000) [\emph{grifo nosso}].

Caso o texto citado traga algum tipo de destaque dado pelo autor do trecho, devemos usar a expressão \emph{grifo do autor, entre colchetes}.

A verdadeira felicidade é encontrada nos pequenos detalhes que vão se somando \emph{dia após dia} de convivência com o ser amado \cite[p. 12]{guerrero2000} [\emph{grifo do autor}].

Quando o texto citado for composto por informações orais obtidas em aulas, palestras, debates, comunicações, etc. deve-se, entre parênteses, colocar a observação \emph{informação oral}, mencionando-se os dados disponíveis em nota de rodapé:

Eichenberg constatou que, na costa do Rio Grande do Sul, especialmente no litoral norte, há a presença abundante de conformes fecais, especialmente nos meses do verão (informação oral). Essa presença tem causado graves transtornos a todos os veranistas.

Se for o caso de se fazer menção a algo contido em \emph{polígrafos}, \emph{apostilas} ou quaisquer materiais avulsos, faz-se a indicação do nome do autor, quando for possível sua identificação, acrescentando-se a observação \emph{`polígrafo'}, \emph{`material de propaganda'}, \emph{`panfleto'}, etc. Procede-se da mesma forma com relação à data. Indica-se, se houver, caso contrário, registra-se s.d. (sem data).

\begin{table}[h]
    \caption{Observação quanto às aspas}
    \begin{center}
        \begin{tabular}{ p{\textwidth} }
            \hline
            As ações longas (mais de três linhas) \emph{não recebem aspas} letra é menor (tamanho 10) do que a do texto (tamanho 12). \\
            \hline
             \\
            \hline
        \end{tabular}
    \end{center}
    \legend{Fonte: \cite[p. 100]{furaste2000}}
    \label{tab:tipos-citacao-longa}
\end{table}

\section{Citação de citação}

Se, num Trabalho, for feita uma citação de alguma passagem \emph{já citada em outra obra}, a autoria deve ser referenciada pelo \emph{sobrenome do autor original} seguido da palavra latina \emph{apud} (que significa \emph{segundo}, \emph{conforme}, \emph{de acordo com}) \emph{e o sobrenome do autor da obra consultada}. Dessa última, faz-se a referência completa (NBR6O23).

``O sistema consiste em colocar o recém-nascido no berço, ao lado da mãe, logo após o parto ou algumas horas depois, durante a estada de ambos na maternidade'' \apud[p. 79]{harunari}{guaragna1992}.

Temos aí palavras de Harunari que foram citadas por Guaranga e que estão sendo utilizadas, agora, no meu trabalho.

\emph{Fonte}: FURASTÉ, Pedro Augusto. Normas Técnicas para o Trabalho Científico: explicitação das normas da ABNT. Porto Alegre: [s.n.], 2002. p. 49-56.
