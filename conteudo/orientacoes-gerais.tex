\chapter{Orientações gerais}

Este capítulo tem o objetivo de descrever os detalhes necessários à correta formatação do documento. As informações aqui apresentadas devem ser suficientes para formatar corretamente o documento com qualquer ferramenta de edição.

Este documento foi criado utilizando estilos. Observe isso com atenção.

Os capítulos são sempre iniciados em uma nova folha. O título do capítulo é formatado todo em letras maiúsculas, com fonte Helvetica (ou semelhante) tamanho 16 pt, em negrito. Para os capítulos não-numerados (Listas, Resumo, Abstract, Referências, etc.), o título é centralizado na linha. Para os numerados, é alinhado à esquerda, precedido do respectivo número. Deve-se deixar 90 pt de espaçamento anterior (ou seja, distância da margem superior) e 42 pt de espaçamento posterior (espaço até o início do texto ou primeira subdivisão). O texto deve ser escrito em espaçamento simples, com observância de 6 pt de espaçamento em relação ao parágrafo seguinte.

\section{Sobre os títulos e capítulos}

As demais subdivisões do texto (seções, subseções, etc.) são formatadas com o título alinhado sempre à esquerda, precedido da respectiva numeração. Esta é formada pela união dos números relativos a cada nível de subdivisão, separados por pontos. Não se inclui um ponto no final.

São permitidas subdivisões até o 5º. nível (onde o capítulo é o 1º. nível), porém no sumário inclui-se somente os títulos até o nível 3. Os parâmetros para formatação dos títulos e espaçamentos nos diversos níveis de subdivisões são apresentados na Tabela \ref{tab:formatacao-subdivisao}.

\begin{table}[h]
    \caption{Parâmetros para formatação das subdivisões do texto}
    \begin{center}
        \begin{tabular}{ c | c | c | c | c }
            \hline
            Nível & Tamanho & Estilo & Esp. Antes & Esp. Depois \\
            \hline
            1 (capítulo) & 16 pt & negrito, maiúsc. & 90 pt & 42 pt \\
            \hline
            2 (seção) & 14 pt & negrito & 18 pt & 9 pt \\
            \hline
            3 (subseção) & 12 pt & negrito & 12 pt & 6 pt \\
            \hline
            4 & 12 pt & itálico & 12 pt & 6 pt \\
            \hline
            5 & 12 pt & normal & 12 pt & 6 pt \\
            \hline
        \end{tabular}
    \end{center}
    \legend{Fonte: FURASTÉ, 2002. p. 49-56.}
    \label{tab:formatacao-subdivisao}
\end{table}

\subsection{Sobre o sumário}

Relaciona as principais divisões e seções do texto, na mesma ordem em que nele se sucedem, indicando, ainda, as respectivas páginas iniciais. O sumário deverá ser localizado imediatamente após as folhas de rosto, catalogação na publicação, dedicatórias e agradecimentos. Para maiores detalhes, ver a norma NBR-6027 da ABNT (1989b).

Os títulos das subdivisões do texto são apresentados em fonte tamanho 12 pt, com as seguintes variações de estilo:

\begin{itemize}
    \item Capítulos: fonte Helvetica, negrito, todas em maiúsculas
    \item Seções: fonte Times, negrito
    \item Subseções: fonte Times, normal
\end{itemize}

Não devem ser incluídos títulos das seções de 4o. e 5o. nível, nem o detalhamento dos Apêndices e/ou Anexos.

No caso de o trabalho ser apresentado em mais de um volume, cada um deve conter o sumário geral da obra, bem como seu próprio sumário, ocupando páginas consecutivas.

\subsubsection{Sobre a lista de abreviaturas e siglas}

Todas as abreviaturas e siglas devem ser ordenadas alfabeticamente e seguidas de seus respectivos significados. Um exemplo pode ser visualizado no início deste documento.

\subsubsection{Sobre a lista de símbolos}

Semelhante à lista de abreviaturas e siglas, os símbolos utilizados no documento devem ser apresentados na ordem em que nele aparecem, acompanhados de seus respectivos significados.

\subsubsection{Sobre as listas de figuras e de tabelas}

Separadamente para as Figuras e Tabelas, devem ser relacionadas as ilustrações na ordem em que aparecem no texto, indicando, para cada uma, o seu número, legenda e página onde se encontra.

\section{Numeração das páginas}

Os números de página são colocados na margem superior do documento, a 2 cm da borda superior do papel, alinhados à margem externa do texto. Por margem externa entende-se a margem direita nas páginas ímpares e a esquerda nas páginas pares. Quando o documento é produzido somente-frente, utiliza-se sempre a margem direita para a numeração.

Todas as páginas do documento, a partir da folha de rosto, são contadas, mas a numeração só é mostrada a partir do primeiro capítulo de texto propriamente dito (ou seja, normalmente a Introdução). Assim, as primeiras páginas não devem apresentar numeração.
