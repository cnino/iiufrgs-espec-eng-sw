\begin{abstract}

Consiste na apresentação clara e concisa dos pontos relevantes do trabalho, de maneira a permitir ao leitor saber da conveniência ou não da sua leitura na íntegra. É redigido pelo autor, em português e em inglês, em páginas distintas, antecedendo a introdução. Cada um ocupará no máximo 1 folha, e poderão ter \emph{até 500 palavras}. Para maiores informações com relação à redação consultar a NBR-6028 da ABNT (1990).

Quanto ao estilo, o resumo deve ser composto por uma seqüência de frases completas e não por uma enumeração de tópicos; a primeira frase deverá ser significativa, explicando o tema principal do documento. Na redação, dar preferência ao uso da terceira pessoa do singular e do verbo na voz ativa. Após o resumo e o abstract devem constar palavras-chave relativas aos assuntos da monografia, em português e inglês respectivamente. Estas são alinhadas na margem inferior do documento.

A ABNT define resumo como: ``[\ldots] seqüência de frases concisas e objetivas e não de uma simples enumeração de tópicos, não ultrapassando 500 palavras, seguido logo abaixo, das palavras representativas do conteúdo do trabalho, isto é, palavras-chave e/ou descritores, conforme a NBR 6028.''

Este item serve para informar o conteúdo do trabalho, orientando assim, o leitor na certeza da continuidade, ou desistência  da leitura do mesmo.
\end{abstract}
